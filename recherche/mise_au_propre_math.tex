\documentclass[11pt, oneside]{article}   	% use "amsart" instead of "article" for AMSLaTeX format
\usepackage{geometry}                		% See geometry.pdf to learn the layout options. There are lots.
\geometry{letterpaper}                   		% ... or a4paper or a5paper or ... 
%\geometry{landscape}                		% Activate for rotated page geometry
%\usepackage[parfill]{parskip}    		% Activate to begin paragraphs with an empty line rather than an indent
\usepackage{graphicx}				% Use pdf, png, jpg, or eps§ with pdflatex; use eps in DVI mode
								% TeX will automatically convert eps --> pdf in pdflatex		
\usepackage{amssymb}
\usepackage{amsmath}

%SetFonts

%SetFonts


\title{Brief Article}
\author{The Author}
%\date{}							% Activate to display a given date or no date

\begin{document}
%\maketitle
%\section{}
%\subsection{}


On cherche \`a maximiser le score avec

\begin{equation}
	\mathit{score} = \sum_{c \; \in \; C_{\mathit{prises}}} \mathit{score}_c
\end{equation}

$score_c$ repr\'esente la valeur de la collection $c$.\\

$C_{\mathit{prises}}$ repr\'esente les collections ``prises'', c'est \`a dire les collections dont on a photographi\'e tous les points.

\begin{equation}
	C_{\mathit{prises}} = \{ c \ | \ c  \in  C \;�\mathit{et} \ \forall p \in P_{c} \ \exists (x, s, t) \in X \ \mathit{tq} \ x = p \}
\end{equation}

o\`u

$C$ : l'ensemble des collections

$P_{c}$ : les points de la collection $c$ 

$X$ : les points captur\'es \\

D\'efinissons $X$, l'ensemble des points captur\'es 

\begin{equation}
	X = \{ p \ | \ p \ \in \ \underset{c \ \in \ C}\cup P_{c} \ \mathit{et} \ p \;� \mathit{''a \; \acute{e}t\acute{e} \;  captur\acute{e}''} \}
\end{equation}

On dit que $p$ ``a \'et\'e captur\'e'' si :

\begin{equation}
	\exists s \in S, \ \exists t \in [0, \, t_{max}[ \ \mathit{tel \, que} \ p = (\varphi_{s, t} + \Delta\varphi_{s, t}, \, \lambda_{s, t} + \Delta\lambda_{s, t})
\end{equation}

\emph{ie.} il existe un satellite $s$ qui a prit $p$ en photo au temps $t$

\end{document}  